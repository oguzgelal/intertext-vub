% !TEX root = ../thesis.tex

\chapter{Solution} \label{solution}


Intertext is a platform based on a straightforward premise; it is a family of applications that can interpret IUIDL, an XML based UI description language, and generate appropriate front-ends for the host platform on the fly. Simply put, a developer wanting to create a front-end application uses a generic backend to generate IUIDL, and serve it from an endpoint, say at \texttt{https://intertext.example.com}. Users who wish to use this application use an Intertext client on their preferred device or environment and visit this domain just like in any web browser. The Intertext client then makes a request to this domain, fetches the IUIDL served by this endpoint and generates the user interface as per the instructions received via IUIDL as shown in Figure ~\ref{fig:how_intertext_works}.


Rather than rendering a simple static view, Intertext performs some tasks such as accepting user input, navigating to different screens, making additional requests to fetch more data, keeping the UI updated and reading and writing some data to users local storage; all of which is again orchestrated based on the instructions received by the backend in IUIDL syntax.


\begin{figure}
  \centering
  \includegraphics[width=13cm]{thesis/paper/images/how_it_works.pdf}
  \caption{A diagram showing the general workings of Intertext}%
  \label{fig:how_intertext_works}%
\end{figure}


Intertext aims to support multiple software clients (currently only the web client is implemented) built natively for various platforms that can interpret IUIDL most appropriately to the host device or platform. For instance, users browsing an Intertext app through a smartphone receive an experience optimised for touch screens, command-line interface client users receive an optimised experience for a text-based interface or a user browsing from a low-end device with limited capabilities use the version optimised for low-performance devices to get a comfortable viewing experience and so on. 

\section{Design Principles} \label{designPrinciples}

In order to address the problems mentioned in the \nameref{problemStatement} section, we adopted and implemented the following design principles.

\subsection{No Foreign Code Execution}

There is no foolproof way of entirely securing a piece of software; even the most mature platforms and operating systems sometimes end up vulnerable to security exploits. However, without the ability to execute code on a platform, it is not directly possible to take advantage of vulnerabilities, even if there is one. Furthermore, the nature of Intertext allowed us to adopt disallowing code execution as a design principle. Intertext clients only accept UIUDL code, which is in XML and is not executable. Should a server send anything else, Intertext clients will simply ignore it. Thus, we can guarantee security for the users, regardless of the platform they are using the Intertext client on.

\subsection{Transparency}

Privacy is a common concern among users; primarily due to the recent scandals and data leaks, people started getting more conscious about their data. There is an increasing demand for users to be more in control and be aware of what is exposed and what is not. In order to address this burning need, we adopted transparency as a design principle. 

The most prominent way of achieving this is to have Intertext clients in control of all interactions with the device and with external sources and keep logs in order to make them transparent to the user. The above-mentioned principle that no foreign code will be executed goes hand-in-hand with achieving this, as it would be unrealistic to expect complete control as it would be a non-deterministic approach. A good analogy would be to think of Intertext clients as an API endpoint that runs on users devices and exposes fundamental interactions with the host device and external networks through an API in a fully controlled manner. An application could \dquote{instruct} the Intertext client via IUIDL to perform some actions, such as making a network request, storing data, and accessing the local storage. Intertext client will then block this action until the user grant permission and keep logs every time before performing an action as shown in Figure \ref{fig:permission_flow}

\begin{figure}
  \centering
  \includegraphics[width=13cm]{thesis/paper/images/permission.pdf}
  \caption{A diagram showing the permission flow}%
  \label{fig:permission_flow}%
\end{figure}


\subsection{Black-Box Components}

Intertext adopts a component-based approach with a set of components that developers can use to build their applications. Each component accepts a set of properties, which gives them certain functionality or appearance. Developers are to use these components through the IUIDL. For example, the IUIDL code shown in Figure \ref{fig:iuidl_buttons} and its output for the Intertext web version shown in Figure \ref{fig:iuidl_buttons_output} shows some of the properties that \texttt{button} component accepts. Layout properties such as  \texttt{marginButton} allows positioning of the buttons to be customised, \texttt{intent} properties gives the button a different look based on the use case, properties like \texttt{disabled} can alter its behaviour and so on.

\begin{figure}
\begin{minipage}{\linewidth}
\begin{lstlisting}[language=xml]
<h3>Buttons</h3>

<grid cols="[1,1]">
  <block>
    <button marginBottom="2">default</button>
    <button marginBottom="2" disabled="true">disabled</button>
  </block>
  <block>
    <button marginBottom="2" intent="default">default</button>
    <button marginBottom="2" intent="primary">primary</button>
    <button marginBottom="2" intent="error">error</button>
    <button marginBottom="2" intent="warning">warning</button>
    <button marginBottom="2" intent="success">success</button>
    <button marginBottom="2" intent="info">info</button>
  </block>
</grid>
\end{lstlisting}
\end{minipage}
\caption{UIUDL that renders buttons in several states}%
\label{fig:iuidl_buttons}%
\end{figure}

\begin{figure}
  \centering
  \includegraphics[width=13cm]{thesis/paper/images/buttons.png}
  \caption{Output of IUIDL in Figure \ref{fig:iuidl_buttons} on Intertext web client}%
  \label{fig:iuidl_buttons_output}%
\end{figure}

However other than properties that components accept, they cannot be modified or altered by the developer. The main motive behind this principle is standardisation. When all Intertext applications use the same set of components, we can control how they look and how they are implemented, and ensure certain behavioural and visual aspects to bring the benefits in order to solve some of the problems mentioned in problem statement.

Consistency is one of the most important benefits that this principle enables. A standard look and feel in components prevents users from getting disoriented across applications, and recognise similar patterns. Another benefit is customisability. It is only possible to create one-size-fits-all themes that applies to every component in every application when all the applications uses components that are built in the same way. Another thing that comes to mind is accessibility. We mentioned in Section \ref{problemStatement} that accessibility is a big issue that not all developers choose to address. This principle allows us to take this responsibility from the developers' hands by providing accessible components as building blocks. 

Last but not least, this principle was crucial to achieve cross-device compatibility. The appearance and behaviour of components differs between implementations of the Intertext clients. For instance, the \texttt{button} component on the web version needs not to be too big as clicks are precise, it requires a hover and focus state. For a touch interface however, it needs to be bigger, and handle the caveat of having less precision by accepting hits on an area around it. Every feature that components have needs to be supported as much as possible on different platforms that has different requirements. Having components with a fixed set of features allowed us to implement them all for different Intertext clients.

\subsection{Shared Syntax}

We designed IUIDL to be a generic markup language, agnostic of any platform or interaction type, and be based purely on XML so it could be consumed by all clients. This approach has many advantages, both for developers and users. It allows developers to create universal applications; once they start serving an Intertext application through an endpoint, any Intertext endpoint can consume the application through that endpoint. Moreover, it helps creating a continual experience for the user; given that there is a shared layout system, UI elements will look and feel the same between Intertext clients.

Also, this approach allows existing Intertext applications to adopt to new Intertext clients as they are built in the future. At the time of writing this paper, only the Intertext web client is ready, but as mentioned in the \nameref{futureWork} section, more Intertext clients are on the way. Thanks to this principle, Intertext clients will have immediate availability for the upcoming Intertext clients.

\section{Intertext UIDL} \label{intertextUIDL}

The Intertext UIDL (IUIDL) is an XML-based markup language that can be used to describe user interfaces. It features a layout system and UI components that could be used as building blocks to put together a user interface. IUIDL is meant to be served from a generic backend, assembled on the backend based on the application logic. It is designed to be unopinionated, as it does not restrict how you assemble it or serve it, or where you serve it from. For instance, below is an example To-do list UI served in IUIDL from a Node.js server that uses Express.js framework, as shown in Figure \ref{fig:todos_js}. Once hit, the endpoint \texttt{/todo} passes the To-do item data to a template file which uses Handlebar as a template engine, and it renders the UI in IUIDL format. The output XML is served from the endpoint.

\begin{figure}
\begin{minipage}{\linewidth}
\begin{lstlisting}[language=javascript]
router.get('/todo', function(req, res, next) {

  const todos = [
    {
      title: 'Buy milk',
      done: false,
    },
    {
      title: 'Call mom',
      done: false,
    },
    {
      title: 'Prepare for presentation',
      done: true,
    },
  ];

  res.render('todo', {
    itemsToDo: todos.filter(item => !item.done),
    itemsDone: todos.filter(item => item.done),
  });
});
\end{lstlisting}
\end{minipage}
\caption{A simple Express endpoint that serves the render output of the Handlebars template shown in Figure \ref{fig:todos_template}}%
\label{fig:todos_js}%
\end{figure}

\begin{figure}
\begin{minipage}{\linewidth}
\begin{lstlisting}[language=xml]
<h3>To do ({{ itemsToDo.length }})</h3>

{{#each itemsToDo}}
  <block intent="default" flexDirection="row" alignItems="center" paddingLeft="4">
    <text flexGrow="1">{{this.title}}</text>
    <button intent="error">Remove</button>
    <button intent="success" marginLeft="2">Done</button>
  </block>
{{/each}}


<h3>Done ({{ itemsDone.length }})</h3>

{{#each itemsDone}}
  <block intent="default" flexDirection="row" alignItems="center" paddingLeft="4">
    <text flexGrow="1">{{this.title}}</text>
    <button intent="error">Remove</button>
  </block>
{{/each}}
\end{lstlisting}
\end{minipage}
\caption{Handlebars template that renders To-do items in IUIDL format}%
\label{fig:todos_template}%
\end{figure}

\begin{figure}
  \centering
  \includegraphics[width=13cm]{thesis/paper/images/todos.png}
  \caption{IUIDL served from the endpoint in Figure \ref{fig:todos_js} rendered on the Intertext web client}%
  \label{fig:todos_output}%
\end{figure}

\subsection{Styling}

IUIDL is agnostic of the styling. There are several \textit{intent}'s that components can accept, which renders the component most appropriately based on its use case. For instance, the Remove buttons in Figure \ref{fig:todos_output} as can be seen as red, and Done buttons green. That is because the \textit{error} intent was given to the Remove button and \textit{success} intent was given to the Done button. However, in this configuration, there is no way to specify how the components look like in particular. The main idea is customisability. Intertext provides themes that users can choose from, and for each theme UI components look and feel differently. Currently, only light and dark themes are available for the web version of Intertext (the dark version of the To-do list UI can be seen in Figure \ref{fig:todos_output_dark}), but as mentioned in the \nameref{futureWork} section, more themes will be made available. Moreover, users will also be able to create custom themes that exactly match their liking.

\begin{figure}
  \centering
  \includegraphics[width=13cm]{thesis/paper/images/todos_dark.png}
  \caption{Dark version of the To-do list UI from Figure \ref{fig:todos_output} rendered on Intertext web client}%
  \label{fig:todos_output_dark}%
\end{figure}

\subsection{Syntax}

As explained in detail in the \nameref{implementation} section, IUIDL gets converted from XML syntax to JSON syntax. However, the conversion to JSON is only an implementation detail that takes place during the rendering phase, and developers do not need to be aware of. XML was chosen at the cost of creating an additional transpilation layer, to achieve optimum developer friendliness. 

When it comes to building UIs, XML and XML-like languages are very common. The main benefit of such languages is that they are easy to read and write, while at the same time they are easy to parse and process as data. IUIDL is meant to be served from a backend, but developers still need to write IUIDL code by hand on the server side. JSON is far from being practical to write in large by hand. The difference could be observed in Figure \ref{fig:syntax_json} and Figure \ref{fig:syntax_xml}, the IUIDL code for a custom 404 page could be seen in both XML and JSON syntax.

\begin{figure}
\begin{minipage}{\linewidth}
\begin{lstlisting}[language=xml]
<block intent="error">
  <h3 intent="error">Not Found</h3>
  <collapse>
    <collapse.handle>
      <text intent="error">
        Show Details
      </text>
    </collapse.handle>
    <p intent="error">... stack trace here ...</p>
  </collapse>
</block>
\end{lstlisting}
\end{minipage}
\caption{The XML representation of the 404 page}%
\label{fig:syntax_xml}%
\end{figure}

\begin{figure}
\begin{minipage}{\linewidth}
\begin{lstlisting}[language=json]
[
    {
        "block": [
            {
                "h3": "Not Found",
                "intent": "error"
            },
            {
                "collapse": [
                    {
                        "p": "... stack trace here ...",
                        "intent": "error"
                    }
                ],
                "handle": [
                    {
                        "text": "\n        Show Details\n      ",
                        "intent": "error"
                    }
                ]
            }
        ],
        "intent": "error"
    }
]
\end{lstlisting}
\end{minipage}
\caption{The JSON representation of the 404 page}%
\label{fig:syntax_json}%
\end{figure}


One issue that raised from the XML syntax was the inability to assign complex children to attributes of a specific tag. As it can be seem from the JSON representation of the 404 page in Figure \ref{fig:syntax_json}, the component \texttt{collapse} requires two sets of children: one for its direct children (specified as \texttt{collapse}), and one for the children to be rendered in its handle (specified as \texttt{handle}). Both \texttt{collapse} and \texttt{handle} properties take other UI components as their children. However, the XML syntax only allows attribute values of a tag to be of type string. In order to solve this problem, we extended XML to create a special syntax to handle such cases. We introduced a special tag name convention: when an XML tree is placed as the children of a component with the tag name being the parent components name concatenated with an attribute name joined together with a dot (\dquote{.}), it gets interpreted as an attribute of the parent, with its value being the children of that XML tree as shown in Figure \ref{fig:complex_attributes}.

\begin{figure}
\begin{minipage}{\linewidth}
\begin{lstlisting}[language=xml]
<parent>
  <parent.complex>
    <text>
      This is under 'complex' attribute of parent
    </text>
  </parent.complex>
  <text>
    This is the real children of parent
  </text>
</parent>
\end{lstlisting}
\end{minipage}
\caption{Complex attributes}%
\label{fig:complex_attributes}%
\end{figure}


\subsection{Terminology}

A last thing to mention in this section is the terminology used in IUIDL. The IUIDL syntax is shared between multiple devices and environments, which includes desktop interfaces as well as touch interfaces, non-graphical user interfaces and more. As mentioned in the \nameref{futureWork} section, as more Intertext clients gets released, the variety will further increase. This brings up a problem with the terminology. For instance, the \texttt{click} that normally would make sense for a desktop environment does not make sense for touch interfaces, as the interaction for a touch screen interface would be \texttt{touch} or \texttt{tap}. In the case of a command-line interface, the primary interaction is focusing on an item and hitting the \texttt{enter} key. 

Our immediate reaction to solve this problem was to create custom jargon that is agnostic of device/interaction type, just like IUIDL is by nature, and find terms for every component/interaction that applies globally. For instance, instead of \texttt{button} we used \texttt{callToAction}, and instead of \texttt{onClick} we used \texttt{onPrimaryInteraction}. However we later abandoned this approach for the sake of developer friendliness. As we custom-named components and their respective actions, we realised that their names were getting very unusual and non-familiar, to a point where it was very difficult to even recognise them. Should we have chosen the global naming convention, we would have introduced a steep learning curve; developers would have needed to get familiar with a long list of terms that they have never heard of before, and they would strictly need to follow the documentation while building Intertext applications.

Instead, we decided to adopt the terminology for desktop/web environments. The rationale behind this decision was that these environments have been around for a very long time, that the terms used to build desktop applications and web applications has a strong recognition among many developers. Also, terms used for desktop environments were mostly generic enough that it was easy to represent on other environments. For instance, a \texttt{button} is a component to be interacted with that performs an action, and \texttt{click} is the method of primary interaction with that component. We can easily take this as a given to replicate it on any environment and create an optimised version taking the limitations of the host platforms in mind.

\section{Intertext Clients} \label{intertextClients}

Intertext clients are the user facing products of Intertext, that build the bridge between the user and servers that serve IUIDL. They are meant to be implemented natively for the host platform in the most optimal way possible. 

For example, on mobile platforms such as iOS and Andriod, UI elements would be larger and more suitable for touch interactions. Primary interaction would be translated as \texttt{tap} while secondary interactions, if any, would be translated as \texttt{tap and hold}. The layout would adjust itself based on the screen size. The client would be implemented with either native technologies such as Swift/Objective-C for iOS and Kotlin/Java for Android, or with hybrid technologies that gets compiled into native code, such as React Native or Flutter for maximum responsiveness and efficiency. 

At the time of writing this paper, only a web client has been implemented. A command-line client, native clients for iOS and Android, and desktop clients for Mac OS, Windows and several Linux distributions are among the ones that are planned, more could be found in \nameref{futureWork} section.

Intertext clients consists of two main parts; components and commands. Each Intertext client implement these separately, based on the host platform and its capabilities. Each client also implements the  their own local storage, state management strategy, and communication technique with the server.

\subsection{Components}

Components are anything that has a visual representation. All component stem from a generic IUIDL definition, and is interpreted based on the host platform requirements. They can be organisational or presentational. Organisational components are the layout components, they are used to position presentational components on the screen. Intertext implements a version of css flexbox specification as a layout system, which could be used to implement responsive interfaces for all screen sizes. For non browser-based platforms, we use the popular Yoga layout engine, which is a standalone implementation of the css flexbox specification in C++, allowing us to target almost any platform. More details on the layout system is given in \nameref{implementation} section.

Most presentational Intertext components share a concept, \texttt{intent}, which is a way to communicate the intention of the UI element based on its use-case. Intertext clients renders the UI elements differently based on its intent, allowing a way for the developer to convey an intention of an UI element to the user without being able to intercept with the styles. For instance, an error message could be shown in a block component that has the intent \texttt{error}, and it will be rendered with a red background, indicating to the user that it is an error message. Figure \ref{fig:intents} shows an example of intents on block component for Intertext web client.

\begin{figure}
  \centering
  \includegraphics[width=13cm]{thesis/paper/images/intents.png}
  \caption{Intents for Block components}%
  \label{fig:intents}%
\end{figure}

\subsection{Commands}

Commands are IUIDL statements that are used to instruct Intertext clients to take an action. These actions can be triggered upon interaction with some components (such as clicking a button or submitting a form), a life-cycle event, with a timeout, or on page load. Commands are very primitive by design, they are not meant to build application logic with. They are merely for asking the Intertext client to communicate with the server, store something, read something that is stored, or retrieve some user data. Command system is designed in order to ensure Intertext client is aware of what it is being asked to do, so that it can control and restrict the entire flow, ask for permission from the user when necessary, or simply refuse to take an action if needed. 

Figure \ref{fig:ex_cmd_post_timeout} shows an example of how a network request action can be executed upon a delay, and Figure \ref{fig:ex_cmd_form} shows how a form can be submitted with reference to input fields. More on commands will be on \nameref{implementation} section.

\begin{figure}
\begin{minipage}{\linewidth}
\begin{lstlisting}[language=xml]
<timeout delay="1000">
  <request endpoint="/refresh"></request>
</timeout>
\end{lstlisting}
\end{minipage}
\caption{Post command on Timeout}%
\label{fig:ex_cmd_post_timeout}%
\end{figure}

\begin{figure}
\begin{minipage}{\linewidth}
\begin{lstlisting}[language=xml]
<input name="item_title"></input>
<input name="item_description"></input>
<button>
  <text>Submit</text>
  <button.onClick>
    <request endpoint="/items/save"></request>
  </button.onClick>
</button>
\end{lstlisting}
\end{minipage}
\caption{Form command with reference on server side to internal input values}%
\label{fig:ex_cmd_form}%
\end{figure}

\begin{figure}
\begin{minipage}{\linewidth}
\begin{lstlisting}[language=javascript]
router.get('/signup', async (req, res, next) => {
  
  const currentStep = req.body.state?.current_step ?? 0
  const currentInputState = req.body.inputState
  const previousInputState = req.body.state.input_state
  
  // check if user finished the form
  if (currentStep == LAST_STEP) {
    
    // execute signup logic
    const error = await signupUser(req.body.state?)
    
    // render success or error page based on
    // the signup result
    res.render(error
      ? 'signup_form_error'
      : 'signup_form_success',
    { error });
  
  } else {
    
    // render signup form
    res.render('signup_form', {
      // increment the current step
      step: currentStep + 1
      // merge current form data with previous
      inputState: merge(
        previousInputState,
        currentInputState
      )
    });
  }
});
\end{lstlisting}
\end{minipage}
\caption{Handling a sign-up flow with multiple steps}%
\label{fig:state_signup_progress_js}%
\end{figure}

\subsection{State Management}

Intertext clients offer basic state management, making it possible to serve stateful Intertext applications using serverless/stateless backends. Front-end state can be fully driven from the backend via IUIDL. Intertext client offers two types of storage, persisted and volatile. 

Volatile storage is kept in the memory and it is persisted across the session. The purpose of this storage is to keep temporary values, such as users' progress in a long form distributed across multiple pages. Intertext clients passes the entire application state on the volatile storage to the backend on every request, allowing backend to build logic around the current state of the front-end application, and serve the UI accordingly. Figure \ref{fig:state_signup_progress_js} and Figure \ref{fig:state_signup_progress_xml} show this through an hypothetical sign up form with steps distributed across multiple steps. Figure \ref{fig:state_signup_progress_js}, we see the pseudo-code implementation of the stateless backend for this scenario. It can be seen that the current step user is at, and the input state data collected thus far is kept in the volatile state on the front-end. The state gets passed on to the backend on every request made to the \texttt{/signup} endpoint. We retrieve the current step, and serve different responses accordingly. If user completed the last step, then we execute our application logic (which in this case is signing user up), and render the result page. Otherwise, we increment the current step, combine the form data collected thus far, and render the form template with these. On the form template in Figure \ref{fig:state_signup_progress_xml}, we use the \texttt{<state>} block to instruct the Intertext client to record these new set of data to the front-end, so it could be passed on to the backend on the next request. Moreover, we render the correct form UI based on the step user is currently at.

Another type of storage is the persisted storage. As the name suggests, values stored this way is persisted across sessions. Unlike the volatile storage, in which the data is kept in the memory, persisted storage gets stored locally using the storage option available to the host platform. By default, data available in the persisted storage is not passed on to the backend in every request automatically, but can be configured to do so.

In order to protect user privacy, Intertext clients block cross-origin storage reads/writes, preventing users to be tracked across Intertext applications. In another words, state management is bound to the origin. Lets say \texttt{sub1.example.com} wrote data to the storage. This data can only be read or manipulated by the very same domain. \texttt{sub2.example.com} or \newline\texttt{sub.example2.com} or any other domain will not have access to this data. While this behaviour protects users from cross-origin tracking, it may also inconvenience some users, especially in scenarios such as services that shares the same user login that is distributed across multiple domains/subdomains. As mentioned further in the \nameref{futureWork} section, we plan to add a feature where Intertext clients would ask for user permission to share data between domains/subdomains instead of directly blocking it.

\begin{figure}
\begin{minipage}{\linewidth}
\begin{lstlisting}[language=xml]
<!-- signup_form -->

<!-- set front-end state to read on later requests  -->
<state key="current_step">{{ step }}</state>
<state key="input_state">{{ inputState }}</state>

{{#ifEquals step 1}}
  <!-- form elements for step 1 -->
{{/ifEquals}}

{{#ifEquals step 2}}
  <!-- form elements for step 2 -->
{{/ifEquals}}

<!-- ... -->
\end{lstlisting}

\begin{lstlisting}[language=xml]
<!-- signup_form_success -->

<h1 intent="success">Sign Up Successful!</h1>
<p intent="success">Please check your verification email</p>
\end{lstlisting}

\begin{lstlisting}[language=xml]
<!-- signup_form_error -->

<h1 intent="error">Something went wrong</h1>
<p intent="error">{{error}}</p>
\end{lstlisting}

\end{minipage}
\caption{Template files for multi-step sign up flow in Figure \ref{fig:state_signup_progress_js}}%
\label{fig:state_signup_progress_xml}%
\end{figure}

